\documentclass[12pt]{letter}
\usepackage[utf8]{inputenc}

\signature{din Vitaly}
\date{Oktober, 18}

\begin{document}

\begin{letter}{\bf Ett mejl till vännen som kommer första gången till din hemstad}
\opening{Hej, Patrick!}

Det är kul du ska komma till Sankt Petersburg nästa vecka! Bland annat den är min hemstad!

Du har haft flera frågor om din visit.

Första, ett kollektivtrafiksystem. Det finns spårvagnar, tunnelbana, bussar och trådbussar i Sankt Peterburg. I T-banan kan du köpa en resekort som fungerar för alla kollektivtransportmedel. Det kostar ungefär 500 rubel per en månad.

Andra, du har frågat hur kan du betala. Ja, du kan ibland betala med kort. Men jag rekommenderar att ha också kontanter med dig. Du kan växla euro till rubel i bankerna. Växla inte i andra platser! Den kan vara mycket dyrt! Växla bara i bankerna!

Din tredja fråga var om sevärdheter. Det är komplicerad att ge en kort svar till den frågan. Sankt Peterburg är en förutvarande ryska huvudstad. Det finns många sevardheter där. Du kan söka på Internet. Men jag tror du borde inte sakna Eremitaget, Peterhof och uppståndelsekyrkan (Kyrkan på blodet).

Tyvärr, jag kan inte träffa dig i Sankt Peterburg. Jag är i Helsingfors nu. Jag studera hård svenska. Jag preparerar  inför en allmämspråk exam.  Men nu finns det ett snabb tåg mellan Sankt Petersburg och Helsingfors. Jag ska vara lycklig om kan du komma tillbacka till Stockholm genom Helsingfors. Vi kan möta och dricka öl tillsammans!


\closing{Bästa hälsningar,}
\end{letter}

%% Mejl 2
\signature{Vitaly Repin}
\address{Dålig hotell\\ Dåliggatan 13 \\ Phatthaya\\ Thailand}

\begin{letter}{\bf Reklamation till dålig hotell}
\opening{Hej,}

I september (14.09 - 18.09) bodde jag i ditt hotel. Det var inte bra upplevelsen för mig.

Första, ingen träffade mig i flygplatsen som blev lovad. Jag måste själv hitta på taxi och förklara en destination till en chaufför . Det var svårt eftersom chauffören talade inte engelska eller svenska - han talade bara thailändska. Jag valde ditt hotel eftersom du har en svenska språkiga  personal!

Andra, varje morgon väckte jag klockan 7 eftersom du renoverade ditt hotell! Det är bra idé men varför måste jag  höra dina arbetare?

Tredje, därför du renoverade luktade illa min våning och rum! Jag hade en huvudvärk och måste drika huvudvärktabletter!

Jag kräver ersättningen för den problemen jag har beskrev!

\closing{MVH,}
\end{letter}

%% Mejl 3
\signature{Vitaly Repin, ingenör, Esbo}
\address{HBL:s redaktion\\ Mannerheimvägen 18\\ Helsingfors\\ Finland}

\begin{letter}{\bf Insändare: lika lön åt alla}
\opening{\textit{Vill våra fackförenings ledare bo i Sovjetunionen eller Nordkorea?}}

I går (17.10) läste jag i HBL en artikel av herr Stenen (Fackförenings ledare) med rubriken: "Löner måsta vara lika åt alla!". Jag trodde det finns ingenting och ingen från våra fackföreningar som kan förvänar mig. Men herr Stenen har lyckats!  Jag gratulerar honom! Det är inte lätt.

Herr Stenen föreslår att betala en samma lön åt alla. Jag vill inte kommentera om den galen idé. Men jag måste påminner det fanns inte lika praktik i utdöd Sovjetunionen och det finns inte ens lik praktik i modern Nordkorea. Sovjetiska regimer försökte att regulera löner och att ha lite skillnad mellan olika löner.

Det var mycket intressant och nyttigt experimentet.  Det resultatet var inte ovanlig. Om lön är samma åt alla människor motiverar inte att arbeta bättre, studera och bli bättre. Det resultaten är ekonomikatastrof!

Vill fackförenings ledare identiska resultaten för Finland?

\closing{\hspace{1cm}}

\end{letter}

\end{document}
